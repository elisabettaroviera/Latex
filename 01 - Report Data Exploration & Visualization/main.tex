\documentclass[10pt]{extarticle}

% Lingua e matematica
\usepackage[english]{babel}
\usepackage{amsmath,amssymb,amsthm}

% Grafica e tabelle
\usepackage{graphicx}
\usepackage{subcaption}
\usepackage{float}
\usepackage{booktabs}
\usepackage{multirow}
\usepackage{siunitx}
\usepackage{wrapfig}
\usepackage{placeins}


% Codice (opzionale)
\usepackage{listings}
\usepackage{xcolor}
\definecolor{codegreen}{rgb}{0,0.6,0}
\definecolor{codegray}{rgb}{0.5,0.5,0.5}
\definecolor{codepurple}{rgb}{0.58,0,0.82}
\definecolor{backcolour}{rgb}{0.98,0.98,0.98}
\lstdefinestyle{mystyle}{
  backgroundcolor=\color{backcolour},
  commentstyle=\color{codegreen},
  keywordstyle=\color{magenta},
  numberstyle=\tiny\color{codegray},
  stringstyle=\color{codepurple},
  basicstyle=\ttfamily\footnotesize,
  breaklines=true, keepspaces=true, numbers=none, tabsize=2
}
\lstset{style=mystyle}

% Margini
\usepackage[margin=0.6in]{geometry}
\usepackage{ragged2e}
\usepackage{enumitem}
%\usepackage[utf8]{inputenc} % per caratteri UTF-8 nei .tex e nel .bib
\usepackage[backend=biber,style=ieee,sorting=none,maxbibnames=99]{biblatex}
\ExecuteBibliographyOptions{doi=true,url=true,isbn=false}
\addbibresource{references.bib}
\usepackage{csquotes}       % consigliato da biblatex (evita warning e parse strani)
\usepackage{pifont}
\newcommand{\good}[1]{\textcolor{teal!70!black}{\ding{51}~#1}}



% Bibliografia (biblatex + biber)
\usepackage[backend=biber,style=ieee]{biblatex}
\addbibresource{references.bib}

% TOC: includi anche \paragraph e numerali
\setcounter{secnumdepth}{2}
\setcounter{tocdepth}{2}

% Hyperref (sempre *dopo* gli altri pacchetti)
\usepackage[colorlinks=true,linkcolor=blue,citecolor=teal,urlcolor=magenta]{hyperref}


\begin{document}

% Custom header without a separate titlepage
\noindent
\begin{minipage}{0.3\textwidth}
    \includegraphics[width=1.3\linewidth]{Figures/polito_logo_2021_blu.jpg}
\end{minipage}
\hfill
\begin{minipage}{0.68\textwidth}
    \raggedleft
    {\LARGE \textbf{Politecnico di Torino}}\\[0.2cm]
    {\large Master's Degree in Mathematical Engineering}\\[0.7cm]
    {\large \textbf{Material for Thesis}}\\[0.2cm]
    {\large b-- Dataset Exploration and Visualization \\ Intra and Inter Analysis}\\[0.7cm]
    \begin{tabular}{rl}
        Elisabetta Roviera & \texttt{s328422} \\
    \end{tabular}
\end{minipage}

\vspace{1cm}
\hrule
\vspace{0.5cm}

\tableofcontents

\vspace{0.5cm}
\hrule
\vspace{1cm}


% Main content begins here, on the same page
\justifying

\begin{abstract}
{\color{red} SCRIVERE ABSTRACT + METTERE LINK NOTEBOOK}

\end{abstract}


\section{Intra-Dataset Exploration and Visualization}
\label{sec:intra_dataset}

Exploratory Data Analysis (EDA) represents a fundamental step in DNA methylation studies, 
particularly in cancer epigenomics, where alterations in global methylation levels, 
sample-level variability, and locus-specific instability often precede detectable 
phenotypic changes. Before applying any preprocessing, filtering, bias correction, or 
machine-learning models, each dataset must be evaluated independently in order to assess 
its internal structure, detect technical anomalies, and identify biologically meaningful 
patterns.

In the context of breast cancer, epigenetic deregulation manifests through both global and 
local phenomena: genome-wide hypomethylation, focal hypermethylation of tumor-suppressor 
regions, and the presence of pre-neoplastic ``field defects'' in histologically normal 
tissues close to tumors \cite{ehrlich2002, bird2002, teschendorff2016}. 
The primary biological goal of this thesis—\emph{to identify DNA methylation alterations 
in normal tissues and evaluate whether they may contribute to the transition toward 
tumorigenesis}—motivates this detailed intra-dataset investigation.

This section analyzes each dataset (GSE69914 [\ref{subsec:gse69914}], GSE225845, GSE287331) separately and 
systematically, focusing on:
missingness and quality control, global methylation distributions, sample-level summaries, 
CpG-level instability, correlation structure, dimensionality reduction, and genome-context 
coverage.

% ============================================
% 2.1 GSE69914
% ============================================
\subsection{Dataset GSE69914}
\label{subsec:gse69914}

\paragraph{Dataset overview}
\label{par:gse69914_overview}

The GSE69914 dataset provides genome-wide DNA methylation profiles generated using the 
Illumina Infinium HumanMethylation450 BeadChip on bisulfite-converted DNA, covering more than 
480{,}000 CpG sites. The series, contributed by Teschendorff and Widschwendter and publicly 
released on June 18, 2015, is available on GEO under accession 
\href{https://www.ncbi.nlm.nih.gov/geo/query/acc.cgi?acc=GSE69914}{GSE69914}.

In total, the dataset includes \textbf{407 breast tissue samples} spanning the three tissue states central 
to this thesis:
\begin{itemize}[label=-]
    \item 50 Normal samples, representing the physiological methylation baseline;
    \item 42 Tumor-adjacent samples, reflecting early epigenetic alterations in proximity to the tumor;
    \item 305 Tumor samples, providing the malignant endpoint of the Normal~$\rightarrow$~Adjacent~$\rightarrow$~Tumor axis.
\end{itemize}

A small subset consists of BRCA1-related samples (7 normal carriers and 3 tumors). BRCA1 encodes a 
key DNA-repair protein, and pathogenic variants substantially increase lifetime breast and ovarian 
cancer risk \cite{brca_factsheet}, adding a hereditary-risk dimension to the cohort.

The final working matrix therefore contains 407 tissue samples and 485{,}512 CpG probes, offering a 
structured and biologically coherent dataset for intra-dataset exploratory analyses.






% ============================================
% Missingness
% ============================================
\paragraph{Missingness and data quality}
\label{par:gse69914_missingness}

An assessment of missing values across all samples and CpG probes confirmed that the 
GSE69914 methylation matrix contains \emph{no missing entries} of any kind. The dataset is 
therefore complete, technically clean, and well suited for direct exploratory analysis and 
visualization without requiring imputation or preliminary data recovery steps.




% ============================================
% Beta distributions
% ============================================

\paragraph{Global methylation distributions}
\label{par:gse69914_global_meth}

To characterise the overall methylation landscape of the dataset, I first examined the 
distribution of $\beta$-values (fractional methylation in $[0,1]$).  
The group-wise mean density curve (Fig.~\ref{fig:gse69914_group_mean_density}) displays the 
characteristic \emph{bimodal} profile of Illumina 450k arrays, with peaks near unmethylated 
($\beta \approx 0$) and fully methylated ($\beta \approx 1$) CpG sites. This pattern reflects 
the underlying biology of CpG regulation, where many loci tend to be either transcriptionally 
active (hypomethylated) or repressed (hypermethylated) \cite{bird2002, maksimovic2012}.

When comparing tissue groups, a clear gradient emerges.  
Tumor samples show a slightly flatter high-$\beta$ peak and a broader low-$\beta$ tail, 
consistent with the well-described phenomenon of \textbf{global hypomethylation} and increased 
heterogeneity in cancer \cite{ehrlich2002}. Tumor-adjacent samples lie between Normal and 
Tumor, suggesting early epigenetic drift and subtle field effects occurring in histologically 
non-neoplastic tissue \cite{teschendorff2016}. Normal samples exhibit the sharpest and most 
stable bimodality, representing the expected physiological baseline.

\begin{figure}[H]
    \centering
    \includegraphics[width=0.5\linewidth]{Figures/GSE69914_04_beta_density_group_mean.pdf}
    \caption{Group-wise mean $\beta$-value density.}
    \label{fig:gse69914_group_mean_density}
\end{figure}



% ============================================
% Outlier burden
% ============================================

\paragraph{CpG-level instability and recurrent outliers}
\label{par:gse69914_outliers}

To characterise locus-specific instability, I examined CpG sites that show the strongest 
deviations across samples. The heatmap of the top outlier loci (Fig.~\ref{fig:gse69914_outliers_heatmap}) indicates that epigenetic disruption is 
\emph{not uniform}: specific CpG sites and specific samples display extreme deviations, 
rather than a diffuse genome-wide shift. Moreover, both directions of alteration are 
present — focal \textbf{hypermethylation} (very high $\beta$) and focal 
\textbf{hypomethylation} (very low $\beta$).  
In cancer biology, hypermethylation can silence tumor-suppressor regions, whereas 
hypomethylation can derepress oncogenic pathways and weaken genomic stability, reflecting 
the classic “too much and too little methylation’’ behaviour of tumor genomes 
\cite{ehrlich2002}.

Complementary $\Delta\beta$ distributions comparing Tumor and Adjacent tissues against Normal 
(Fig.~\ref{fig:gse69914_outliers_deltabeta}) show a clear shift toward 
hypomethylation in Tumor samples and a subtler but detectable drift in Adjacent tissues.  
This provides evidence that epigenetic alterations emerge early in histologically normal 
tissue located near the tumor, supporting the field-cancerization model 
\cite{teschendorff2016}.

\begin{figure}[H]
    \centering
    \begin{subfigure}[t]{0.48\linewidth}
        \centering
        \includegraphics[width=\linewidth]{Figures/GSE69914_11_heatmap_top_outlier_cpgs_numbered.pdf}
        \caption{Top outlier CpG loci.}
        \label{fig:gse69914_outliers_heatmap}
    \end{subfigure}
    \hfill
    \begin{subfigure}[t]{0.48\linewidth}
        \centering
        \includegraphics[width=\linewidth]{Figures/GSE69914_12_delta_beta_overlaid.pdf}
        \caption{$\Delta\beta$ distributions (Tumor/Adjacent $-$ Normal).}
        \label{fig:gse69914_outliers_deltabeta}
    \end{subfigure}
    \caption{CpG-level instability: heatmap of top outlier CpGs (left) and 
    corresponding $\Delta\beta$ distributions (right).}
    \label{fig:gse69914_outliers_combined}
\end{figure}


% ============================================
% Variance, correlation, moments
% ============================================
\paragraph{Variance and correlation structure}
\label{par:gse69914_variance_corr}
To quantify within-group epigenetic variability, I examined the distribution of CpG-wise
variance across samples. The density curves (Fig.~\ref{fig:gse69914_var_corr_variance}) show that
Tumor samples have markedly higher variance, reflecting increased epigenetic instability.
In contrast, Normal and Tumor-adjacent samples display almost identical variance
distributions that are tightly concentrated near zero.  
This indicates that, at the level of CpG-wise variability, Adjacent tissues do not yet
exhibit detectable divergence from Normal samples, despite being histologically proximate
to the tumor.

A complementary perspective is provided by the sample correlation heatmap 
(Fig.~\ref{fig:gse69914_var_corr}b), which shows the expected high level of 
pairwise similarity across all samples. This behaviour is typical of genome-wide DNA 
methylation data, where a large fraction of CpG sites is stable across individuals, 
resulting in consistently strong correlations \cite{teschendorff2013}.  

\begin{figure}[H]
    \centering
    \begin{subfigure}[t]{0.48\linewidth}
        \centering
        \includegraphics[width=\linewidth]{Figures/GSE69914_13_intragroup_variance.pdf}
        \caption{CpG-wise variance distributions.}
        \label{fig:gse69914_var_corr_variance}
    \end{subfigure}
    \hfill
    \begin{subfigure}[t]{0.48\linewidth}
        \centering
        \includegraphics[width=\linewidth]{Figures/GSE69914_14_sample_correlation_heatmap.pdf}
        \caption{Sample correlation heatmap.}
        \label{fig:gse69914_var_corr_correlation}
    \end{subfigure}
    \caption{Variance and correlation structure across Normal, Adjacent, and Tumor samples.}
    \label{fig:gse69914_var_corr}
\end{figure}


% ============================================
% DR: PCA, tSNE, UMAP
% ============================================
\subsubsection*{Low-dimensional embeddings}

Dimensionality reduction (Figures~\ref{fig:gse69914_pca}--\ref{fig:gse69914_umap}) reveals 
a clear separation between Normal and Tumor samples.  
Adjacent samples consistently populate a transitional manifold between these two groups, 
supporting the hypothesis of a continuous epigenetic gradient rather than an abrupt 
transition \cite{teschendorff2016}.

\begin{figure}[H]
    \centering
    \includegraphics[width=0.55\linewidth]{Figures/GSE69914_16_pca.pdf}
    \caption{PCA projection.}
    \label{fig:gse69914_pca}
\end{figure}

\begin{figure}[H]
    \centering
    \includegraphics[width=0.55\linewidth]{Figures/GSE69914_17_tsne.pdf}
    \caption{t-SNE embedding.}
    \label{fig:gse69914_tsne}
\end{figure}

\begin{figure}[H]
    \centering
    \includegraphics[width=0.55\linewidth]{Figures/GSE69914_18_umap.pdf}
    \caption{UMAP embedding.}
    \label{fig:gse69914_umap}
\end{figure}

% ============================================
% Manifest coverage
% ============================================
\subsubsection*{Genome annotation coverage}

Coverage across genomic contexts (islands, shores, shelves) matches the expected 
distribution for the 450K array, confirming correct manifest matching.

\begin{figure}[H]
    \centering
    \includegraphics[width=0.55\linewidth]{Figures/GSE287331_19_coverage_by_cpg_context.pdf}
    \caption{CpG context coverage.}
    \label{fig:gse69914_manifest}
\end{figure}

% ============================================
% Dynamic network proxies
% ============================================
\subsubsection*{Dynamic-network proxies}

Although not a full Dynamic Network Biomarker (DNB) analysis, the degree-based proxies 
(Figures~\ref{fig:gse69914_dnb_degree}--\ref{fig:gse69914_dnb_adjacent}) show enhanced 
network instability in Tumors and early deviations in Adjacent samples.

\begin{figure}[H]
    \centering
    \includegraphics[width=0.55\linewidth]{Figures/GSE287331_20_dynamic_network_degree_proxy.pdf}
    \caption{Gene-level CpG degree (proxy).}
    \label{fig:gse69914_dnb_degree}
\end{figure}

\begin{figure}[H]
    \centering
    \includegraphics[width=0.55\linewidth]{Figures/GSE287331_21_dnb_network_tumor_minus_normal.pdf}
    \caption{DNB-like network --- Tumor vs Normal.}
    \label{fig:gse69914_dnb_tumor}
\end{figure}

\begin{figure}[H]
    \centering
    \includegraphics[width=0.55\linewidth]{Figures/GSE287331_22_dnb_network_adjacent_minus_normal.pdf}
    \caption{DNB-like network --- Adjacent vs Normal.}
    \label{fig:gse69914_dnb_adjacent}
\end{figure}

% ============================================
% Summary
% ============================================
\subsubsection*{Summary}

The GSE69914 dataset exhibits a highly coherent and biologically interpretable structure. 
Across all analyses—global distributions, variability, outlier patterns, and embeddings—
Tumor samples consistently show global hypomethylation, increased heterogeneity, and focal 
instability. Adjacent samples systematically occupy an intermediate position, consistent 
with early epigenetic alterations and the field-defect model.  
No technical issues were detected, making GSE69914 an excellent baseline dataset for 
subsequent comparative and modeling analyses.


\subsection{GSE225845}

\subsection{GSE287331}



% ============================
% Inter Dataset Analysis
% ============================
\section{Inter Dataset Analysis}



\printbibliography
\end{document}
