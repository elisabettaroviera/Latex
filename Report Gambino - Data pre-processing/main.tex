\documentclass[10pt]{extarticle}

% Lingua e matematica
\usepackage[english]{babel}
\usepackage{amsmath,amssymb,amsthm}

% Grafica e tabelle
\usepackage{graphicx}
\usepackage{subcaption}
\usepackage{float}
\usepackage{booktabs}
\usepackage{multirow}
\usepackage{siunitx}
\usepackage{wrapfig}
\usepackage{placeins}


% Codice (opzionale)
\usepackage{listings}
\usepackage{xcolor}
\definecolor{codegreen}{rgb}{0,0.6,0}
\definecolor{codegray}{rgb}{0.5,0.5,0.5}
\definecolor{codepurple}{rgb}{0.58,0,0.82}
\definecolor{backcolour}{rgb}{0.98,0.98,0.98}
\lstdefinestyle{mystyle}{
  backgroundcolor=\color{backcolour},
  commentstyle=\color{codegreen},
  keywordstyle=\color{magenta},
  numberstyle=\tiny\color{codegray},
  stringstyle=\color{codepurple},
  basicstyle=\ttfamily\footnotesize,
  breaklines=true, keepspaces=true, numbers=none, tabsize=2
}
\lstset{style=mystyle}

% Margini
\usepackage[margin=0.6in]{geometry}
\usepackage{ragged2e}
\usepackage{enumitem}
%\usepackage[utf8]{inputenc} % per caratteri UTF-8 nei .tex e nel .bib
\usepackage[backend=biber,style=ieee,sorting=none,maxbibnames=99]{biblatex}
\ExecuteBibliographyOptions{doi=true,url=true,isbn=false}
\addbibresource{references.bib}
\usepackage{csquotes}       % consigliato da biblatex (evita warning e parse strani)
\usepackage{pifont}
\newcommand{\good}[1]{\textcolor{teal!70!black}{\ding{51}~#1}}



% Bibliografia (biblatex + biber)
\usepackage[backend=biber,style=ieee]{biblatex}
\addbibresource{references.bib}

% TOC: includi anche \paragraph e numerali
\setcounter{secnumdepth}{2}
\setcounter{tocdepth}{2}

% Hyperref (sempre *dopo* gli altri pacchetti)
\usepackage[colorlinks=true,linkcolor=blue,citecolor=teal,urlcolor=magenta]{hyperref}


\begin{document}

% Custom header without a separate titlepage
\noindent
\begin{minipage}{0.3\textwidth}
    \includegraphics[width=1.3\linewidth]{Figures/polito_logo_2021_blu.jpg}
\end{minipage}
\hfill
\begin{minipage}{0.68\textwidth}
    \raggedleft
    {\LARGE \textbf{Politecnico di Torino}}\\[0.2cm]
    {\large Laurea Magistrale in Ingegneria Matematica}\\[0.7cm]
    {\large \textbf{Analisi Comparativa e Strategie Future per lo Studio Epigenetico del Carcinoma Mammario}}\\[0.2cm]
    {\large Revisione dei Lavori di Tesi \\ Data Pre-Processing Pipeline in GSE69914}\\[0.7cm]
    \begin{tabular}{rl}
        Eleonora Guarnaschelli \\
        Elisabetta Roviera
    \end{tabular}
\end{minipage}

\vspace{1cm}
\hrule
\vspace{0.5cm}

\tableofcontents

\vspace{0.5cm}
\hrule
\vspace{1cm}


% Main content begins here, on the same page
\justifying

\begin{abstract}
{\color{red}
\textbf{MANCA L'ABSTRACT, LO SCRIVO DOPO.}\\
}
\end{abstract}

\paragraph{Costruzione del dataset.}\label{sec:data_storage}


Abbiamo acquisito il dataset \texttt{GSE69914}~\cite{GSE69914} in formato \texttt{.txt} contenente i valori di metilazione ($\beta$-values) e ottenuto le etichette di classe direttamente da \texttt{GEOparse} per garantirne la massima correttezza.  
La matrice è stata successivamente convertita in un file compresso \texttt{.parquet} con righe corrispondenti ai campioni e colonne alle sonde CpG.  
Infine, abbiamo eseguito i controlli di qualità per verificare l’assenza di valori mancanti e la correttezza del range dei $\beta$-values.


\section{Correzione del bias tra sonde Infinium I e II}\label{sec:infinium_bias}

I dati grezzi di intensità di metilazione del dataset sono stati pre-processati dagli autori originali utilizzando il pacchetto \texttt{minfi} (v1.8.9) e la normalizzazione \texttt{BMIQ} (v1.4), come riportato in~\cite{GSE69914}.  
Questa pipeline include la \textbf{correzione del bias tra le sonde Infinium di tipo~I e tipo~II}, assicurando che le distribuzioni dei valori di metilazione ($\beta$-values) dei due design risultino comparabili prima delle analisi successive.

Abbiamo verificato in modo indipendente che tale correzione fosse effettivamente applicata.  
A tal fine, le distribuzioni dei $\beta$-values sono state stratificate per tipo di sonda (Type~I vs.~Type~II) utilizzando il manifest ufficiale di Illumina~450K ottenuto dal pacchetto Bioconductor~\cite{Illumina450kAnno}.  
Seguendo l’approccio diagnostico proposto da Teschendorff et~al.~\cite{Teschendorff2013BMIQ}, abbiamo generato due grafici di controllo.
\begin{enumerate}
    \item Le distribuzioni di densità dei $\beta$-values per le sonde Type~I e Type~II (\autoref{fig:type_bias_density}) mostrano un’elevata sovrapposizione, senza evidenza di bias dovuto al design;
    \item Il Q--Q plot che confronta i quantili delle sonde Type~II rispetto alle Type~I (\autoref{fig:type_bias_qq}) presenta un allineamento quasi diagonale, confermando l’efficacia della correzione.
\end{enumerate}

Il pattern osservato dopo la normalizzazione riproduce fedelmente i risultati descritti da Teschendorff~et~al.~\cite{Teschendorff2013BMIQ} e da Wang~et~al.~\cite{Wang2015Normalization450K}, che hanno dimostrato come la procedura BMIQ riduca sostanzialmente il bias tra sonde Infinium di tipo~I e~II, producendo distribuzioni sovrapposte e relazioni quantiliche bilanciate.  
\good{Pertanto, la nostra verifica conferma che il dataset \textbf{GSE69914} ha subito una corretta normalizzazione BMIQ e non presenta bias residui tra le sonde Infinium~I/II prima dell’analisi.}

\begin{figure}[H]
    \centering
    \begin{subfigure}[t]{0.49\textwidth}
        \centering
        \includegraphics[height=6cm]{Figures/type_bias_density.pdf}
        \caption{Distribuzione dei $\beta$-values per tipo di sonda (Type~I vs.~Type~II).  
        Le curve quasi sovrapposte indicano una correzione efficace del bias.}
        \label{fig:type_bias_density}
    \end{subfigure}
    \hfill
    \begin{subfigure}[t]{0.49\textwidth}
        \centering
        \includegraphics[height=6cm]{Figures/type_bias_qqplot.pdf}
        \caption{Q--Q plot tra i quantili delle sonde Type~II e Type~I.  
        L’allineamento alla diagonale ($y = x$) conferma la correzione del bias di design.}
        \label{fig:type_bias_qq}
    \end{subfigure}
    \caption{Valutazione diagnostica della correzione del bias tra sonde Infinium Type~I e~II nel dataset \textbf{GSE69914}.  
    Le distribuzioni coerenti dei $\beta$-values e dei quantili dimostrano l’efficacia della normalizzazione BMIQ applicata.}
    \label{fig:type_bias_combined}
\end{figure}

\section{Filtraggio tecnico}\label{sec:technical_filtering}

Il filtraggio tecnico ha lo scopo di rimuovere le sonde inaffidabili o affette da artefatti biologici o di piattaforma prima delle fasi di normalizzazione e modellazione statistica.  
Questo passaggio consente di ridurre il rumore, migliorare la riproducibilità delle analisi e mantenere esclusivamente i siti CpG ad alta affidabilità.

\paragraph{Rimozione delle sonde affette da artefatti tecnici.}
Abbiamo applicato diversi insiemi di filtri consolidati presenti in letteratura per eliminare sonde problematiche o non univocamente mappate.  
In particolare, sono stati rimossi:
\begin{itemize}[label=-]
  \item \textbf{SNP-affected probes:} sonde contenenti varianti comuni nel sito CpG, nella base di estensione o all’interno della sequenza della sonda~\cite{zhou2016,pidsley2016};
  \item \textbf{Sonde cross-reattive:} sonde con ibridazione multipla o mappatura non specifica~\cite{chen2013,mccartney2016,zhou2016};
  \item \textbf{Maschere di design o piattaforma:} flag \texttt{MASK\_*} relativi a errori di mappatura, SNP adiacenti, sonde non-CpG e sonde su cromosomi sessuali opzionali~\cite{zhou2016};
  \item \textbf{Controlli gerarchici Naeem (450K):} esclusione di sonde multi-mappate, ripetute o affette da varianti strutturali (INDEL) e SNP interferenti~\cite{naeem2014}.
\end{itemize}

La \autoref{tab:technical_filtering} sintetizza le principali categorie di artefatti tecnici considerate e le corrispondenti fonti bibliografiche utilizzate per la loro rimozione: Naeem~et~al.~2014~\cite{naeem2014}, Chen~et~al.~2013~\cite{chen2013}, Pidsley~et~al.~2016~\cite{pidsley2016}, Zhou~et~al.~2016~\cite{zhou2016} e McCartney~et~al.~2016~\cite{mccartney2016}.  
\good{Complessivamente, sono state rimosse \textbf{225.426 sonde CpG}.}


\begin{table}[H]
\small
\caption{Categorie tecniche coperte da ciascuna risorsa di filtraggio.}
\centerline{
\begin{tabular}{lccccc}
\toprule
\textbf{Categoria} & \textbf{Naeem} & \textbf{Chen} & \textbf{Pidsley} & \textbf{Zhou} & \textbf{McCartney} \\
\midrule
Ibridazione multipla / multi-mapping & Sì  & Sì & Sì & \texttt{MASK\_mapping} & Liste 2–3 \\
SNP al sito CpG / base di estensione & Sì  & -- & Sì & \texttt{MASK\_snp5, MASK\_extBase} & -- \\
SNP adiacenti tollerati & Sì & -- & -- & -- & -- \\
INDEL / varianti strutturali & Sì  & -- & -- & -- & -- \\
Sonde non-CpG & -- & -- & Sì & \texttt{MASK\_nonCG} & Lista 3 \\
\bottomrule
\end{tabular}
\label{tab:technical_filtering}
}
\end{table}


\paragraph{Filtraggio basato su annotazione.}
Abbiamo successivamente verificato la coerenza del dataset con il file di manifest ufficiale di Illumina  
(\href{https://emea.support.illumina.com/array/array_kits/infinium_humanmethylation450_beadchip_kit/downloads.html?utm_source=chatgpt.com/}{\textit{HumanMethylation450 v1.2 Manifest File}}), assicurando il mantenimento esclusivo delle sonde valide e ben caratterizzate.  
Sono state eliminate sonde non-CpG (identificativi con prefisso ``ch'').

Questo passaggio garantisce la consistenza tra i dati sperimentali e l’annotazione ufficiale Illumina, prevenendo disallineamenti genomici nelle analisi successive.  
Seguendo le raccomandazioni di Zhou~et~al.~\cite{zhou2016} e Pidsley~et~al.~\cite{pidsley2016}. \good{Sono state rimosse ulteriori \textbf{875 sonde non-CpG}.}

\section{Filtraggio delle CpG invarianti}\label{sec:invariant_filtering}

I siti CpG che mostrano una variabilità di metilazione minima tra i campioni non forniscono informazioni discriminanti e possono aumentare inutilmente il carico di test multipli.  
Seguendo l’approccio empirico proposto da Edgar~\textit{et~al.}~\cite{Edgar2017}, abbiamo identificato le sonde a bassa dispersione utilizzando l’intervallo inter-decile dei valori di metilazione, definito come:
$r_{\beta} = \mathrm{P90}(\beta) - \mathrm{P10}(\beta)$,
una misura robusta e resistente agli outlier della variabilità di metilazione.

\paragraph{Analisi di variabilità sull’intero dataset.}
Considerando tutti i tipi di tessuto, la distribuzione di $r_{\beta}$ risulta fortemente asimmetrica a destra, con la maggior parte dei loci che mostrano una variabilità limitata (\autoref{fig:beta_range_all}).  
Abbiamo valutato tre soglie candidate ($r_{\beta} < 0.01$, $0.02$, $0.05$), ottenendo:
\good{$r_{\beta} <$ 0.01: 1.782~CpG rimosse},  
\good{$r_{\beta} <$ 0.02: 28.569},  
\good{$r_{\beta} <$ 0.05: 79.433}.

\paragraph{Sottoinsieme Normal e Adjacent.}
Poiché l’obiettivo finale è identificare le CpG informative per distinguere i tessuti \textbf{normali} (label~0) da quelli \textbf{adiacenti} (label~1), abbiamo applicato il filtraggio in modo specifico a questo sottoinsieme.  
Come atteso, limitando l’analisi a tessuti non tumorali la dispersione complessiva diminuisce (\autoref{fig:beta_range_subset}), con i seguenti risultati:
\good{$r_{\beta} <$ 0.01: 3.972~CpG rimosse},  
\good{$r_{\beta} <$ 0.02: 40.342},  
\good{$r_{\beta} <$ 0.05: 102.473}.

Questo conferma che i campioni tumorali contribuiscono in modo predominante alla variabilità globale della metilazione, in accordo con quanto riportato da Hansen~et~al.~\cite{Hansen2011}, che descrivono un’aumentata variabilità stocastica nei tessuti neoplastici rispetto a quelli normali.  
\good{Abbiamo pertanto adottato come soglia di riferimento il valore proposto da Edgar~\textit{et~al.}~\cite{Edgar2017}, $r_{\beta} < 0.05$, applicandolo al sottoinsieme Normal--Adjacent, ottenendo una matrice finale di \textbf{99 campioni $\times$ 156.740 CpG}.}  
Le CpG al di sotto di tale soglia nel sottoinsieme ristretto sono state rimosse anche dal dataset completo.  

Nel caso in cui analisi successive evidenzino che questa scelta risulti eccessivamente permissiva — portando all’esclusione di un numero rilevante di loci potenzialmente informativi — il criterio verrà rivalutato in senso più conservativo (ad esempio soglie di $0.02$ o $0.01$).  
Questo approccio garantisce che vengano scartati solo i siti con metilazione stabile nei tessuti non tumorali, preservando invece le CpG con possibile rilevanza biologica per la distinzione tra tessuti Normal e Adjacent.

\begin{figure}[H]
    \centering
    \begin{subfigure}[t]{0.49\textwidth}
        \centering
        \includegraphics[height=6.2cm]{Figures/beta_range_distribution_study.png}
        \caption{Distribuzione dell’intervallo inter-decile $r_{\beta}$ considerando l’intera coorte (Normal, Adjacent e Tumor).  
        La distribuzione è asimmetrica, con la maggior parte delle CpG a bassa variabilità.}
        \label{fig:beta_range_all}
    \end{subfigure}
    \hfill
    \begin{subfigure}[t]{0.49\textwidth}
        \centering
        \includegraphics[height=6.2cm]{Figures/beta_range_distribution_study_label013.png}
        \caption{Distribuzione dell’intervallo inter-decile $r_{\beta}$ limitata ai tessuti Normal e Adjacent.  
        L’esclusione dei campioni tumorali riduce la dispersione complessiva.}
        \label{fig:beta_range_subset}
    \end{subfigure}
    \caption{Distribuzione dell’intervallo inter-decile dei valori di metilazione ($r_{\beta} = \mathrm{P90} - \mathrm{P10}$) nei siti CpG.  
    L’analisi evidenzia una minore variabilità nei tessuti non tumorali, supportando l’uso di $r_{\beta}<0.05$ come criterio di filtraggio per il sottoinsieme Normal--Adjacent.}
    \label{fig:beta_range_compare}
\end{figure}




\printbibliography
\end{document}
